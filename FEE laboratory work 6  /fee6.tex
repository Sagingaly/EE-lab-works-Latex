\documentclass{article}
\usepackage{graphicx}
\usepackage{afterpage}
\usepackage{setspace}

\begin{document}

\begin{titlepage}
    \centering
    \includegraphics[width=12cm]{images/logo.jpg}
    
    \vspace{0cm} % Add vertical space
    {\fontsize{40}{48}\selectfont Факультет Информационных систем} % Large font size and the text you mentioned
    
    \vspace{0.1cm} % Add vertical space
    
    \begin{center} % Center the following text
        Faculty of Information Technology\\
        Department of Electrical Engineering and Computer Science
    \end{center}
    
    \vfill % Fill the vertical space
    
    
    {\LARGE\bfseries Laboratory Work \#6}\\
    {\LARGE\bfseries Building RLC circuits and work with oscilliscope}
    \vspace{4cm} % Add vertical space
    
    \begin{flushright}
        Done by: Sagingaly Meldeshuly \\
        Checked by: Raushan Amanzholova \\
        November 2023
    \end{flushright}
    
    \vfill % Center the page number vertically
    \begin{center}
        2023, Almaty
    \end{center}
    
\end{titlepage}


\begin{flushleft}
\textbf{Purpose of the work laboratory work:}
\end{flushleft}
\begin{enumerate}
    \i
tem Build and analysis RLC circuit.
    \item work with an oscilloscope, tale measurements
    \item to draw graphs of measurements
\end{enumerate}

\begin{flushleft}
\textbf{Brief theory}
\end{flushleft}

\begin{flushleft}
Electric circuits that consist of only one resistor and one energy storage element (an inductor or a capacitor) are termed as first-order circuits. Their response to a constant input is essential in understanding their transient and steady-state behavior.
\end[]
\begin{flushleft}
\begin{itemize}
    \item Build and Analyze RLC Circuit:
The first objective is to design and construct an RLC circuit in series. RLC circuits are essential components in electronics, and understanding their behavior is crucial. By building the circuit, students will get hands-on experience with the components and the circuit's overall configuration. Additionally, this exercise will help in understanding the principles of electrical impedance and resonance in RLC circuits..

    \item Work with an Oscilloscope and Take Measurements:
The second objective focuses on the practical use of an oscilloscope, a valuable instrument in electronics. Students will learn how to connect the oscilloscope to the RLC circuit to visualize and measure various electrical parameters such as voltage, current, and phase relationships. This experience is essential for understanding how to troubleshoot and analyze electronic circuits and signals in the real world.
\end{itemize}
\end{flushleft}

\newpage
\begin{flushleft}
\textbf{In lab tasks}
\end{flushleft}

\begin{flushleft}
    1.Build the RLC circuit in series and get transient response by oscilloscope. What kind of response it is? a) R= 100 \Omega\ b) R= 200 \Omega, c) R= 300 \Omega
\end{flushleft}

\begin{flushleft}
    \centering
    \includegraphics[width=12cm]{images/Снимок экрана 2023-11-06 111752.png}
\end{flushleft}

\begin{flushleft}
    \centering
    \includegraphics[width=12cm]{images/Снимок экрана 2023-11-06 115115.png}
\end{flushleft}

\begin{flushleft}
    \centering
    \includegraphics[width=14cm]{images/fee.jpeg}
\end{flushleft}

\begin{flushleft}
    \centering
    \includegraphics[width=14cm]{images/fee 2.jpeg}
\end{flushleft}

\begin{flushleft}
    \centering
    \includegraphics[width=15cm]{images/fee 3.jpeg}
\end{flushleft}



\begin{flushleft}
\noindent \textbf{Conclusion}
In summary, this laboratory work provided students with practical experience in working with RLC circuits and oscilloscopes, enabling them to understand electrical impedance, resonance, and measurement analysis. These skills are fundamental for students in electronics and electrical engineering, preparing them for real-world applications and problem-solving in electronic systems.
\end{flushleft}
\end{document}




\documentclass[a4paper]{article}
\usepackage{setspace}
\usepackage[14pt]{extsizes} % setting font size
\usepackage[utf8]{inputenc} % settnig encoding
\usepackage[russian, english]{babel}
\usepackage{setspace}
\usepackage{amsmath}
\usepackage{graphicx}
\usepackage{caption}
\usepackage{tabto}
\usepackage{float}
\usepackage{parskip}
\usepackage[
    left=20mm, top=15mm, right=15mm, bottom=15mm, nohead, footskip=10mm
]{geometry}



\usepackage{times}
\begin{document} 
 

\begin{center}
    \begin{figure}
        \centering
        \includegraphics[width = 18cm]{images/}
    \end{figure}
    
    \large{
        School of Information Technology and Engineering
    }\\
    
    \hfill \break
    \hfill \break
    \hfill \break
    \hfill \break
    \hfill \break
    

    \large{
        \textbf{
            Laboratory work №5\break
            Build and analysis RLC circuit
        }\\
    }\\
    
    \hfill \break
    \hfill \break
    \hfill \break
    \hfill \break
    \hfill \break
    \hfill \break
    
    \begin{flushright}
        Done by: \textbf{Sagingaly Meldeshuly}\break
        Checked by: \textbf{Raushan Amanzholova}
    \end{flushright}
    
    \hfill \break
    
\end{center}

\hfill \break
\hfill \break

\begin{center} 
    Almaty 2023
\end{center}

\thispagestyle{empty}
 


\newpage
\begin{flushleft}
    \large
    \textbf{
        First-order R-L and R-C circuits
    }
    \break

    \textbf{Purpose of a laboratory work:}\\
    - Build and analysis RLC circuit. \break
    - work with an oscilliscope, take measurements \break
    - to draw graphs of measurements \break
    \hfill \break

\begin{flushleft}
\begin{itemize}
    \item Build and Analyze RLC Circuit:
The first objective is to design and construct an RLC circuit in series. RLC circuits are essential components in electronics, and understanding their behavior is crucial. By building the circuit, students will get hands-on experience with the components and the circuit's overall configuration. Additionally, this exercise will help in understanding the principles of electrical impedance and resonance in RLC circuits..

    \item Work with an Oscilloscope and Take Measurements:
The second objective focuses on the practical use of an oscilloscope, a valuable instrument in electronics. Students will learn how to connect the oscilloscope to the RLC circuit to visualize and measure various electrical parameters such as voltage, current, and phase relationships. This experience is essential for understanding how to troubleshoot and analyze electronic circuits and signals in the real world.
\end{itemize}
\end{flushleft}

    \textbf{
        Practice task: \break
        Pre-Lab
    } \\
    \quad 1. Build the RLC circuit and get transient response by oscilliscope. What kind of response it is? a) R=100 {\Omega};b) R=200{\omega}; c) R=300{\omega}\\
    \begin{figure}[H]
        \centering
        \includegraphics[width = 18cm]{
            images/1-task.jpg
        }
    \end{figure}

    $V_c (0^-) = V_c (0^-) = 0v; $ \quad 
    $V_c (\infty) = 5v$ \break 
    \break
    $R = R_1 = 10k \Omega$ \break
    $\tau = RC = 10 * 10^3 * 1 * 10^-^6 = 0.01 s$ \break
    \break
    $V (t) = V_c (\infty) + e^-^t^/^\tau * (V_c(0) - V_c (\infty))$ \break
    $V (t) = 5 - 5* e ^-^1^0^0^t$ \break
    \hfill

    \quad 2. Switch was closed for a long time. Find the voltage on the capacitor, when switch opens. Find steady state voltage on capacitor after switch was open for a long time. Write solution to the RC circuit, and draw the solution with proper scaling. \\
    
    \begin{figure}[H]
        \centering
        \includegraphics[width = 18cm]{
            images/2-task.jpg
        }
    \end{figure} 

    $V_c (0^-) = V_c (0^-) = 5v; $ \quad
    $V_c (\infty) = 10v$ \break
    \break
    $R = R_1 = 10k \Omega$ \break
    $\tau = RC = 10 * 10^3 * 1 * 10^-^6 = 0.01 s$ \break
    \break
    $V (t) = V_c (\infty) + e^-^t^/^\tau * (V_c(0) - V_c (\infty))$ \break
    $V (t) = 10 + (5-10) * e ^-^1^0^0^t = 10 - 5 * e ^-^1^0^0^t$ \break
    \hfill \break

    \quad 3. Switch was open for a long time. Find $V_c$, when switch closes. Find the $V_c$ after switch was closed for a long time. Find the time constant, and write solution to the RC circuit, and draw the solution with proper scaling. \\
    
    \begin{figure}[H]
        \centering
        \includegraphics[width = 18cm]{
            images/3-task.jpg
        }
    \end{figure} 

    \begin{figure}[H]
        \centering
        \includegraphics[width = 18cm]{
            images/3-task_2.jpg
        }
        \caption{Equivalent circuit when switch was closed}
    \end{figure}

> Anthony Howard:
\begin{flalign}
    \nonumber
        V_c = V = \frac{50}{50+10+20} * 12 = 7.5v  && 
    \end{flalign}
    \begin{flalign}
    \nonumber
        V_c (0^-) = V_c (0^-) = 7.5v &&
    \end{flalign}
    \begin{flalign}
    \nonumber
       R = \frac{R_1 * R_3}{R_1+R_3} = 14.28k \Omega &&
    \end{flalign}
    \begin{flalign}
    \nonumber
       \tau = RC = 14.28 * 10^3 * 1 * 10^-^6 = 0.0143 s &&
    \end{flalign}
    \begin{flalign}
    \nonumber
       V_c(\infty) = \frac{50}{50+20} * 12 = 8.57v &&
    \end{flalign}
    \begin{flalign}
       V (t) = 8.57 + (7.5-8.57) * e ^-^t^/^0^.^0^1^4^3 = 8.57 - 1.07 * e ^-^6^9^.^9^3^t &&
    \nonumber
    \end{flalign}
    \hfill \break

    \quad 4. Read the code on the capacitors, and find value of capacitors. \\
    a) $104 \Rightarrow 10 * 10^4 pF = 10^5 * 10^-^1^2 = 0.1 \mu F$ \break
    b) $474 \Rightarrow 47 * 10^4 pF = 47 * 10^4 * 10^-^1^2 = 470 nF$ \break
    c) $154 \Rightarrow 15 * 10^4 pF = 15 * 10^4 * 10^-^1^2 = 150 nF$ \break

    \textbf{ 
        In-Lab \break
    } \\
    
    \quad 1. Build the RC circuit and get transient response by oscilloscope. Find time constant from oscilloscope. \\
    
    \begin{figure}[H]
        \centering
        \includegraphics[width = 18cm]{
            images/1-Task_Sim.jpg
        }
    \end{figure}

    \begin{figure}[H]
        \centering
        \includegraphics[width = 18cm]{
            images/1-Task_Charging_Graph.jpg
        }
        \caption{Capacitor charging in simulation}
    \end{figure}
    
    \begin{figure}[H]
        \centering
        \includegraphics[width = 18cm]{
            images/1-Task_Discharging_Graph.png
        }
        \caption{Capacitor discharging in experiment}
    \end{figure}
    
    \break
    \quad 3. Build the RC circuit and get transient response by oscilloscope. Find time constant from oscilloscope. \\
    \begin{figure}[H]
        \centering
        \includegraphics[width = 18cm]{
            images/3-Task_Sim.jpg
        }
    \end{figure}
    
    \begin{figure}[H]
        \centering
        \includegraphics[width = 18cm]{
            images/3-Task_Charging_Graph.jpg
        }
        \caption{Capacitor charging in simulation}
    \end{figure}

    \begin{figure}[H]
        \centering
        \includegraphics[width = 18cm]{
            images/3-Task_Discharging_Graph.png
        }
        \caption{Capacitor discharging in experiment}
    \end{figure} \\
    

    \textbf{ 
        Post-Lab \break
    }

    \quad 1.The circuit is at steady state before the switch opens. Find the current $i(t)$ that goes up on $20k\Omega$ resistor for $t > 0$. \\
    \begin{figure}[H]
        \centering
        \includegraphics[width = 18cm]{
            images/Post-Lab_1.jpg
        }
    \end{figure}
    \begin{flalign}
    \nonumber
        V (0^-) = V (0^+) = V_2 = 7v &&
    \end{flalign}
    \begin{flalign}
    \nonumber
       R = \frac{R_1 * R_3}{R_1+R_3} + R_2 = 64.28k \Omega &&
    \end{flalign}
    \begin{flalign}
    \nonumber
       \tau = RC = 64.28 * 10^3 * 3 * 10^-^6 = 0.193 s &&
    \end{flalign}
    \begin{flalign}
    \nonumber
       V(\infty) = \frac{20}{20+50} * 10 = 2.857 v &&
    \end{flalign}
    \begin{flalign}
    \nonumber
       i = \frac{2.857}{20*10^3} = 0.143 mA &&
    \end{flalign}
    \\
    \hfill \break
    
    \quad \textbf{Conclusion:} \\
    \quad In summary, this laboratory work provided students with practical experience in working with RLC circuits and oscilloscopes, enabling them to understand electrical impedance, resonance, and measurement analysis. These skills are fundamental for students in electronics and electrical engineering, preparing them for real-world applications and problem-solving in electronic systems.
\end{flushleft}
\end{document}
