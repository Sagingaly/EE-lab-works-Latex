\documentclass{article}
\usepackage{graphicx}
\usepackage{afterpage}
\usepackage{setspace}

\begin{document}

\begin{titlepage}
    \centering
    \includegraphics[width=12cm]{images/20090516224137!Logotip_KBTU (1) (1).jpg}
    
    \vspace{0cm} % Add vertical space
    {\fontsize{40}{48}\selectfont Факультет Информационных систем} % Large font size and the text you mentioned
    
    \vspace{0.1cm} % Add vertical space
    
    \begin{center} % Center the following text
        Faculty of Information Technology\\
        Department of Electrical Engineering and Computer Science
    \end{center}
    
    \vfill % Fill the vertical space
    
    {\LARGE\bfseries Laboratory Work \#5}\\
    {\LARGE\bfseries First- order RL and RC circuits}
    \vspace{4cm} % Add vertical space
    
    \begin{flushright}
        Done by: Sagingaly Meldeshuly \\
        Checked by: Raushan Amanzholova \\
        September 2023
    \end{flushright}
    
    \vfill % Center the page number vertically
    \begin{center}
        2023, Almaty
    \end{center}
    
\end{titlepage}



% "*" убирает нумерацию в разделе section
% "$ $" в долларах мы пишем математических операций
% " \times " это умножение с крестиком
% P_R_2=$I_R_2 \times V_R_2$
% \[ numerator and denominator
%\frac{a}{b}
%\]
% power ^
% \subsection подтопик
% Omega
% \\ is used to create a new line or break a line. It's similar to pressing "Enter" or "Return" on your keyboard when you're writing a regular text document.
\begin{flushleft}
\textbf{Purpose of the work laboratory work:}
\end{flushleft}
\begin{enumerate}
    \item To identify the First-Order Electric circuit’s response to a Constant Reference Input
    \item To Check Sequential Switching
    \item To Check Stability of First-Order Circuits
\end{enumerate}

\begin{flushleft}
\textbf{Brief theory}
\end{flushleft}

\begin{flushleft}
Electric circuits that consist of only one resistor and one energy storage element (an inductor or a capacitor) are termed as first-order circuits. Their response to a constant input is essential in understanding their transient and steady-state behavior.

\begin{itemize}
    \item First-Order R-L Circuit: When a resistor and inductor are connected in series and are subjected to a step voltage, the current doesn't instantaneously reach its steady-state value due to the presence of the inductor. The inductor opposes sudden changes in current. The resulting current vs. time response can be exponential depending on the values of resistance and inductance.

    \item First-Order R-C Circuit: A resistor-capacitor circuit behaves similarly, but in response to a step voltage, the voltage across the capacitor doesn't change instantaneously. This is because capacitors oppose sudden changes in voltage. The resulting voltage vs. time curve is also typically exponential based on resistance and capacitance values.
\end{itemize}
\end{flushleft}

\newpage
\begin{flushleft}
\textbf{Pre lab tasks}
\end{flushleft}

\begin{flushleft}
    Switch was open for a long time. Find Vc, and when switch closes. Find the Vc after switch was closed for a long time. Find the time constant , and write solution to the RC circuit, and draw the solution with proper scalling. Given R1=10kOhm and C2=1microF and V=5v
\end{flushleft}

\begin{flushleft}
    \centering
    \includegraphics[width=12cm]{images/WhatsApp Image 2023-10-15 at 18.23.28.jpeg}
\end{flushleft}

\begin{flushleft}
    \centering
    \includegraphics[width=12cm]{images/fee1.png}
\end{flushleft}

\begin{flushleft}
    \centering
    \includegraphics[width=12cm]{images/fee2.png}
\end{flushleft}

\begin{flushleft}
    \centering
    \includegraphics[width=12cm]{images/WhatsApp Image 2023-10-15 at 18.50.39.jpeg}
\end{flushleft}


\begin{flushleft}
\section*{104:}
First two numbers: 10 \\
Multiplier: $10^4$ \\
Resulting value: $10 \times 10^4 = 100,000$ pF, which is also 100 nF or 0.1 $\mu$F.

\section*{474:}
First two numbers: 47 \\
Multiplier: $10^4$ \\
Resulting value: $47 \times 10^4 = 470,000$ pF, which is also 470 nF or 0.47 $\mu$F.

\section*{154:}
First two numbers: 15 \\
Multiplier: $10^4$ \\
Resulting value: $15 \times 10^4 = 150,000$ pF, which is also 150 nF or 0.15 $\mu$F.

\section*{Summary:}
The capacitor with ``104'' has a value of 0.1 $\mu$F. \\
The capacitor with ``474'' has a value of 0.47 $\mu$F. \\
The capacitor with ``154'' has a value of 0.15 $\mu$F.
\end{flushleft}

\begin{flushleft}
\noindent \textbf{Conclusion}
{During the laboratory work on "First-order R-L and R-C circuits", we examined the transient response of these circuits when subjected to a constant reference input. Through this investigation, we learned the innate nature of inductors and capacitors in resisting sudden changes, which leads to the characteristic exponential response of first-order circuits.
We also explored the importance of sequential switching. We understood that the order in which components are switched on or off can have significant impacts on the circuit behavior, especially during transients.
\end{flushleft}
\end{document}
